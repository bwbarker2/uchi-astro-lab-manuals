\chapter{Discovering the Hydrogen Atom}

\section{Learning Goals}

\begin{enumerate}
	\item Learn to identify errors in physical models and modify them through inference and observations of physical phenomena
	
	\item Develop methods for testing and working with objects which cannot be directly observed
	
	\item Gain an understanding of the properties of atomic structures and why these differ from what classical mechanics would predict
\end{enumerate}

\section{Lab Team Roles}

Decide which team members will hold each role this week: facilitator, scribe, technician, skeptic.

\section{Observation Experiment} %title likely to change

\textbf{Goal:} When thinking of an atom, we usually tend to imagine a series of electrons orbiting around a nucleus in nice circular orbits. While at first this model might seem unremarkable, the reality is that the physics it describes is a lot weirder than you think! In order to see why this is, lets see if we can reconstruct the model of the atom

%\textbf{Available Equipment:}

%\textbf{Rubric Rows to Focus On:}

\begin{step}
	\item First, lets take the classical planetary model of the atom at face value. Using only classical physics concepts, try and find whats wrong with this model. 
	\begin{itemize} 
		\item Think first about what happens to satellites orbiting Earth as they interact with the atmosphere. What forces are acting on it? What happens to it over time?
	\end{itemize}
	\item Once your group has come up with an answer, describe what happens to the orbiting object in terms of its energy and its position over time. What implications does this have when you apply it to the atom?
	\item After identifying the problem with the model, try and come up with several ideas for a new model which solves the issues you found.
	\begin{itemize}
		\item These models don't have to be too complicated, just try thinking of different way in which the electron might move or behave in relation to the nucleus.
	\end{itemize}
	\item