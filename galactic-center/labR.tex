\chapter{Black Hole at the Galactic Center?}

\section{Mystery at the Center of the Milky Way}
Take a moment to watch the video found in the following link \url{www.astro.ucla.edu/~ghezgroup/gc/animations.html} under the heading \textbf{3D Movie of Stellar Orbits in the Central Parsec}. At first glance the video might not seem all too surprising as having learned about the solar system you likely expect orbiting planets to be a mundane fixture of the universe. However, what if you were to learn that the objects were not planets, but in fact stars and that what you see in the video spans a distance of 3 light years? For comparison, Pluto is only about .0006ly from the sun. In fact, the video you just saw is a visualization of a phenomenon in the center of our galaxy which puzzled astronomers for a long time. As you might have learned all objects exert a gravitational force which is proportional to the mass of the object. For this reason, smaller objects tend to be "pulled in" by larger objects, forming the orbital relationships we see in our daily lives: the moon orbiting the Earth, the Earth orbiting the Sun, and so on. Some of the most massive objects in the universe are stars which is why they tend to form the center of orbital systems. However, given that all the objects in the video were stars, this meant that there had to be a much, much more massive object in the center of our galaxy attracting them, one which seemed to be invisible, save for radio signals coming from the location of the object. There were many theories as to what the object, whose signal is dubbed Sagittarius A*, could  be, but the most compelling one is that it is in fact a super massive black hole (SMBH)

Black holes are some of the most extreme objects in the universe which were first theorized to exist as a result of Einstein's theory of general relativity. In the most basic terms, a black hole is an extremely massive and dense object whose gravitational pull is so strong, that not even light can escape. This fact that light cannot escape from a black hole,  however, makes them incredibly difficult to observe directly. That said, due to the strength of their gravitational pull, black holes can often be detected indirectly based on their influence over nearby objects. In this lab you will examine the gravitational system you saw in the video and you will be able to determine whether the object in the center of the milky way is, in fact, a black hole. First, however, you will learn about the basic laws which govern orbital systems and how they can be applied to determine some of the physical properties of the objects in the system.

\section{goals}
\begin{itemize}
	\item Understand Kepler's laws of planetary motion and be able to use them to extrapolate information about orbital systems
	\item Be able to gather data using a variety of tools and be able to understand the limitations of certain data
	\item Be able to make inferences about physical properties of objects which cannot be directly measured
\end{itemize}

\section{Newtonian Dynamics and Orbital Dynamics Basics}


\section{General Relativity and Schwarzschild Radii}

While Newtonian dynamics is useful for describing most orbital systems, extreme systems or objects such as black holes cannot be fully described without also incorporating general relativity. In particular for this lab we will be using a particular description of the universe in which gravity, rather than being an "attraction" between objects, is actually the result of curved "space-time". To visualize this, imagine space-time as sheet of stretched out fabric. Normally, if you were to try to roll light objects across the sheet they would travel in a straight line. However, if you were to place a large weight in the center, the fabric would "droop" inwards and any object you tried to roll would instead fall inwards towards the depressed region (the following video demonstrates this analogy \url{https://youtu.be/MTY1Kje0yLg}). This is analogous to the effect which gravity has on space time. The key to this description is that anything traveling through space-time will follow this curvature, even if it has no mass such as light. This means that, theoretically, an object can exist which bends space-time so much that not even light can climb back out and escape. Luckily, using the principles of gravitation developed by Newton, we can approximate what such an object might look like. 

\subsection{Steps}
\begin{enumerate}
	\item In newtonian dynamics, the minimum speed an object needs to escape the gravitational pull of an object is given by 
	\begin{equation}\label{gc:eq:escape-speed}
		v_\textrm{escape} = \sqrt{\frac{2 G M}{r}} \,.
	\end{equation}
	where $r$ is the distance from its center of mass $M$. First, manipulate this equation in order to get an expression for $r$ in terms of the other values.
	
	\item If you now plug in the speed of light $c = 2.998 \times 10^{10}$cm/s as the escape velocity into the equation you just derived, you get an expression for what is known as the Schwarszchild radius. The Schwarzschild radius is an estimate of the radius of a black hole and, more importantly, its the minimum size an object of a given mass can be before it becomes a black hole. Using the equation you found, calculate the schwarzchild radii for the following objects.
	\begin{itemize}
		\item One of your group members.
		\item the Earth.
		\item the Sun.
		\item the Solar System.
		\item The Milky Way Galaxy.
	\end{itemize}
\end{enumerate}

