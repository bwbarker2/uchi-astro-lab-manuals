\chapter{Lab Report Format}

In a general sense, the labs should demonstrate Rubric Rows F1 and F2 (see Table~\ref{rubric:f}), in addition to the other rubric rows listed in the lab write-up.

\section{General}

\begin{itemize}
	\item The report should be typed for ease of reading. Text should be double-spaced, and the page margins (including headers and footers) should be approximately $2.5\:$cm, for ease of marking by the grader. Each page should be numbered.
	
	\item The first page should include the title of the lab; lab section day, time, and number; and the names of the members of your lab team.
	
	\item If the rubric row refers to a particular part of your lab report, clearly label that part of the report with that rubric row. For example, you should label the section where you demonstrate uncertainty propagation with ``G2'' if that rubric row is being assessed in that lab.
\end{itemize}

\section{Organizing the report}

If the lab is clearly framed as an observational, testing, or application experiment, you can follow the corresponding rubric for the elements to include in the report (see, respectively, Rubrics B, C, and D in Appendix~\ref{cha:rubrics}).

In general, the report should include the following sections:
\begin{enumerate}
	\item \textbf{Introduction.} A written description of what the lab is designed to investigate and a brief summary of the procedure used. This section should be at least a full paragraph long, and not more than 3 double-space pages.
	
	You don't need to include too much detail here, but it should be a complete and concise description of the purpose and general method used in the lab. Imagine a classmate who hasn't seen the lab writeup asked you, ``what is this lab about? What do you do?'' You should be able to hand them your introduction, and they'd be able to understand the purpose and general structure of the lab. But you need not mention every step and calculation here.
	
	\item \textbf{Analysis and discussion.} For most labs that have more than one part, this should be broken up into parts and labeled in order. This section must include all of the following, in the same order in which these elements appear in the lab instructions.
	\begin{itemize}
		\item Any data that you've collected: tables, figures, measured values, sketches. Whenever possible, include an estimate of the uncertainty of measured values.
		
		\item Any calculations that you perform using your data, and the final results of your calculation. Note that you must show your work in order to demonstrate to the grader that you have actually done it. Even if you're just plugging numbers into an equation, you should write down the equation and all the values that go into it. This includes calculating uncertainty and propagation of uncertainty.
		
		\item If you are using software to perform a calculation, you should explicitly record what you've done. For example, ``Using Excel we fit a straight line to the velocity vs. time graph. The resulting equation is $v = (0.92\:\mathrm{m/s^2}) t + 0.2\:\mathrm{m/s}$.
		
		\item Answers to any questions that appear in the lab handout.
	\end{itemize}

	\item \textbf{Conclusion.} This can be very short, and will generally only require one or two paragraphs. In your conclusion, you should summarize the point of the lab and what you learned, both in the frame of a scientist conducting the experiment (``What did the experiment tell us about the world?'') and in the frame of a student (``What skills or mindsets did I learn?'').
\end{enumerate}

\section{Graphs, Tables, and Figures}

Any graph, table, or figure (a figure is any graphic, for example a sketch) should include a caption describing what it is about and what features are important, or any helpful orientation to it. The reader should be able to understand the basics of what a graph, table, or figure is saying and why it is important without referring to the text. For more examples, see any such element in this lab manual.

Each of these elements has some particular conventions.

\subsection{Tables}

A table is a way to represent tabular data in a quantitative, precise form. Each column in the table should have a heading that describes the quantity name and the unit abbreviation in parentheses. For example, if you are reporting distance in parsecs, then the column heading should be something like ``distance (pc)''. This way, when reporting the distance itself in the column, you do not need to list the unit with every number.

\subsection{Graphs}

A graph is a visual way of representing data. It is helpful for communicating a visual summary of the data and any patterns that are found.

The following are necessary elements of a graph of two-dimensional data (for example, distance vs. time, or current vs. voltage) presented in a scatter plot.

\begin{itemize}
	\item \textbf{Proper axes.} The conventional way of reading a graph is to see how the variable on the vertical axis changes when the variable on the horizontal axis changes. If there are independent and dependent variables, then the independent variable should be along the horizontal axis.
	
	\item \textbf{Axis labels.} The axes should each be labeled with the quantity name and the unit abbreviation in parentheses. For example, if you are plotting distance in parsecs, then the axis label should be something like ``distance (pc)''.
	
	\item \textbf{Uncertainty bars.} If any quantities have an uncertainty, then these should be represented with so-called ``error bars'', along both axes if present. If the uncertainties are smaller than the symbol used for the data points, then this should be explained in the caption.

\end{itemize}
