\chapter{Detecting exoplanets with the transit method}

\setcounter{section}{2}

\section{Scheduling observations}

First, schedule the remote observations. The observations are made at night in Arizona, and they can be scheduled during the day before those observations.

\begin{steps}
%	\item Find the transit calendar in the Lab 3 Files folder in the Files section on Canvas. Pick an exoplanet transit that will be observable sometime during the next two weeks. Note that all times listed here are local to Arizona.

	\item As a group, create a single account that you will all log in to at the DIY Planet Search website at \url{https://waps.cfa.harvard.edu/microobservatory/diy/index.php}.
	
	\item Navigate to About and read the ``About DIY Planet Search'' and ``About MicroObservatory'' sections.
	
%	\item Log into the exoplanet lab website at \url{https://www.cfa.harvard.edu/smgphp/otherworlds/ExoLab/index.html} using the username and password your TA has given you. If you haven't received this yet, skip this section until you receive it.
	
	\item Navigate to ``DIYTools'', watch the ``Schedule Target Tutorial'', and schedule observations of at least two different star systems. For both, choose ``All hours'' and 60 second exposures.
	
\end{steps}

You will only analyze one star system, but it may be cloudy on one night, so observing two different ones makes it more likely you get data to analyze.

