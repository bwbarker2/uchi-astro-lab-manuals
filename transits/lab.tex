\chapter{Detecting exoplanets with the transit method}

\section{Part 1}

See other handout on Canvas for Part 1.

\section{Part 2}

Throughout the upcoming week, you will use    one    of    the    MicroObservatory    telescopes,    built    and    maintained    by    the    Harvard‐Smithsonian    Center    for    Astrophysics    and    located    at    the    Whipple    Observatory    in    Amado,    Arizona    to    take    a    series    of    images    of    a    ``target''    star in order to calculate a light curve for that star, which could be used to learn about the planet(s) orbiting them.    These    images    will    form    the    basis    of    your    subsequent    investigation in the Image Lab on the LSE website.

During this class, you will determine which stars you are imaging and decide among your group who will schedule the observations. After you do this, you will use the LSE website to analyze an example image, so you can analyze your own images later at home for the report.

The report will be due two weeks from today's lab, rather than one week, to account for the extra work at home.

\subsection{Scheduling observations}

Throughout the upcoming week, you will use    one    of    the    MicroObservatory    telescopes,    built    and    maintained    by    the    Harvard‐Smithsonian    Center    for    Astrophysics    and    located    at    the    Whipple    Observatory    in    Amado,    Arizona    to    take    a    series    of    images    of    a    ``target''    star.    These    images    will    form    the    basis    of    your    subsequent    investigation in the Image Lab on the LSE website.

Log on to LSE using the credentials emailed to you and click “Go to the
Lab.”

Click on the “Telescope” tab and go through the sections on the right
side of the page to learn about the remote observatory we are using and
how to schedule observations.
%\textit{Note: you can only schedule observations
%for the current night and the schedule times are Chicago time.}

Next, look at the transit calendar file posted on the canvas course
website, “2019\_April\_May\_transit\_calendar.pdf,” which lists observable
transits by the remote observatory each night. Note, times are for
Tucson, AZ, not Chicago. Work out with your TA and rest of the section a
plan for observing a few transits over the course of the week. Each
transit will require observing for $\sim 1$ hour before and $\sim 1$ hour after the
transit for a total observation time of 4--5 hours. This corresponds to 80--
100 exposures. One possible arrangement would be to have smaller
groups responsible for scheduling different segments of the total
observation, e.g.: group 1 can schedule observing for the 1st hour, group
2 would schedule the second hour, etc\dots{} Make plans for observing a
number of systems throughout the week. If possible, conduct multiple
observations of the same system. \textit{Note: you can only schedule
observations for the current night, so you will need to login and
schedule your observations on the appropriate day of the week.}

Once your observations have been completed, you can analyze them on
your own, but in lab we will practice this analysis with example data.

Click on the “Image Lab” and go through the 9 page tutorial. Practice
your analysis on the “Demo\_Images” for TRES-3. Make sure you subtract
and appropriate Dark Image (note that the image filenames include the
date and time of the exposure).

Once it’s been taken, analyze the data from your scheduled
observations. Be sure to press the “Calculate \& Record” button to save
your results.

Combine your data with the rest of the sections (you might have to re-
do the previous week’s activities) by looking at the “class graph.” \textbf{Save this light curve and turn it in with your write up.}

\section{Report checklist and grading}

Each item below is worth 10 points, and there is an additional 10 points for attendance and participation.

\begin{itemize}
	\item Data table from Part 1.
	
	\item Plots of your light curves for Planet 1, Planet 2, and all of the planets you observed with LSE. 
	
	\item Discuss how different features of the light curve connected to physical
	properties of the orbital system.
	
	\item What is your interpretation of the
	temperatures for the simulated planets (Planet 1 \& 2)?
	
	\item List which scheduled observations your group performed. Were you able to see a light curve from your scheduled observations? Why or why not?
\end{itemize}