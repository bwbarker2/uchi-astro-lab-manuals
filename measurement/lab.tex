\chapter{Observing Falling Filters}

%TODO Notes from 2019 Autumn: add more explicit reference to the error propagation section in appendix.

% TODO Notes from 2018 Autumn, Brent Barker (bbarker@uchicago.edu):
% - add definition of cross-sectional area
% - specify which uncertainty propagation formula to use with converting from diameter to area
% - maybe require the extrapolation to zero area at the end of the write-up

	\begin{quotation}
		\textit{The ability to observe without evaluating is the highest form of intelligence.} \sourceatright{Jiddu Krishnamurti}
	\end{quotation}

While Mr.\ Krishnamuri may be making a stretch with his superlative, it remains true that observing without evaluating is essential for the creation of knowledge.
In our lives, we have bias (conceptions, self-constructed mental models) that we use as our lens to view the world.
These models are based on how each of us were socialized and on our subsequent experience.
To learn and create new knowledge, we must develop skill in observation.
In this lab, we will direct you to make detailed, careful quantitative observations, describe the patterns you find with mathematics, and finally make some wild guesses (``hypotheses'') about a more universal principle that explains this pattern that one could use to make predictions.
Due to time and brain constraints, we will not, in this lab, test those hypotheses.

\section{Learning Goals}

 \begin{itemize}

  \item Become familiar with measurement, uncertainty, and writing lab reports.

  \item Learn how to conduct an observational experiment, including collecting data and analyzing the data to find and describe a pattern quantitatively.

  \item Use measurement uncertainty to describe physical quantities meaningfully.
  
  \item Format a lab report in a helpful way.
 \end{itemize}

\section{The Scientific Cycle\protect\footnote{adapted from \cite{etkina_college_2014}}}

One way of describing science is the process of incrementally improving a shared model of how our universe works. In different fields of science, different methods and cycles are used, so there is no ``One True Scientific Method.'' One can still create a model for the process of science, and we describe here one such cycle, summarized in Figure~\ref{me:fig:isle}.

In this cycle, there are three types of experiments, each one representing a different stage of the scientific effort. One stage, often started when encountering a novel phenomenon, is the \textbf{observational experiment}. This is an experiment that consists of deciding what to observe and how to observe it, collecting data, finding a pattern, and brainstorming possible explanations for what is observed (also called ``hypotheses'').

Once one has some trial explanations, one can test one or more of those with a \textbf{testing experiment}. Here, one designs a new experimental procedure and uses each hypothesis to predict what will happen. Then the prediction is compared to the procedure's outcome. If they are different, then the hypothesis is judged to be not a helpful explanation for that phenomenon. If they are the same, then it is still helpful. Throughout this stage, one may make various assumptions that would need to be validated, as they can effect the prediction or outcome.

Once a hypothesis has been tested enough for people to find it useful, then it can be applied to solve practical problems, or to determine properties of particular situations, in an ``application experiment.''

\begin{figure}
	\centering
	\includegraphics[width=0.7\textwidth]{measurement/islegraphic.png}
	\caption{A model of the process some scientists go through to create knowledge.\cite{etkina_millikan_2015}}\label{me:fig:isle}
\end{figure}

\section{Observation experiment: Observing falling filters}

In today's lab, you will investigate the relationship between the size of coffee filters and how long it takes them to fall. In the first section, you will determine the size of the coffee filters. In the second, you will determine how long each take to fall, controlling for other variables, and then find a mathematical pattern that describes the relationship. Note that this lab does not include any hypothesis testing.

\begin{framed}
	\textbf{Self-assessment:} To help you improve your scientific abilities, we provide you with self-assessment rubrics.
	A rubric is a method of aligning expectations for performance.
	Self-assessment is determining how well you performed a particular task.
	So, these self-assessment rubrics are designed to help you evaluate your performance while you are designing and performing your experiment.
	
	The complete set of rubrics is available in Appendix~\ref{cha:rubrics}.
%	In each lab, your report will be assessed using Rubric F, found in Table~\ref{rubric:f}, as well as 5 additional rubric rows listed in that lab.
%	Each week, read through these and use them to evaluate your work as you design and perform the experiment.
%	Your instructor will use the same rubrics to determine part of your grade for the lab. In particular, each row will be worth 3 possible points (from ``Missing'' being 0 points to ``Adequate'' being 3 points).
\end{framed}	

\textbf{Rubrics to focus on during this experiment:} B5, B7, B8, F1, F2, G1, G2. See Appendix~\ref{cha:rubrics} for details.

\textbf{Available equipment:} several differently-sized coffee filters, meter stick, balance or scale, stopwatch, scratch paper%, camera (on your phone), Computer with ImageJ installed, string

You may want to \textbf{decide on roles} for each group member. Example roles include Facilitator (ensures time and group focus are efficiently used), Scribe (ensures work is recorded), Technician (oversees apparatus assembly, usage), Skeptic (ensures group is questioning itself). Note that each role is responsible for ensuring that the thing happens, rather than necessarily doing it themselves.

\subsection{How big are the filters?}

% DevNote: I decided to not make this a whole application experiment with 2 different methods, for the sake of time. I also removed image analysis with ImageJ, for the same reason. --bbarker, 2018-10-02

\textbf{Goal:} Find the cross-sectional area of each coffee filter and make a determination of that area, including uncertainty in that area, for use in the next section.
%In Stage 1 of the Barker X-Prize, each team has been tasked with determining precisely how big various objects are, including marbles, cotton balls, and coffee filters, including a detailed determination of their uncertainty of these measurements.
 
%\begin{framed}
%	ImageJ is an image analysis program that includes, among other things, the ability to measure lengths, angles, and areas in images, provided that you give it a scale for how long some reference object is in the image.
%\end{framed}

\begin{enumerate}
	\item Review Rubric G (Table~\ref{rubric:g}) and discuss any unclear expectations with your group and the instructor.
	
%	\item Discuss with your group what cross-sectional area means and why it might affect the fall time. Feel free to use your resources (books, internet, etc.) to do this.
	
	\item Brainstorm different methods you could use to determine the cross-sectional area. Feel free to play with the equipment as desired. Here are some things to consider:
	\begin{itemize}
		\item Will you measure the area directly, or will you measure something else and use that to calculate the area?
		
		\item With any method, you will probably make one or more assumptions about the shape of the filter. How valid are those assumptions?
		
		\item For each method you consider, there may be different sources of uncertainty --- the resolution of the measuring devices themselves, how you use them to measure, etc. If there is a source of random uncertainty, then you will need to take several measurements and use Appendix~\ref{unc:random} to determine the uncertainty. The decision of how many measurements to take is a trade-off between increasing precision (decreasing the uncertainty of the mean) and decreasing the time the measurement process takes.
		
		\item If you make a measurement and use that measurement in an equation to find the area, you will need to propagate uncertainty as described in Appendix~\ref{unc:sec:prop}.
	\end{itemize}
	% Come up with two independent methods for determining the cross-sectional area. The purpose is that if you make a mistake or wrong assumption in one method, then the method (hopefully) gives a different result than the other method. For discovering new things, this is one quantitative way of checking your work, since you don't have the answer ahead of time.

	\item Decide on your method and discuss it with an instructor before you begin. They will help increase the chances that your method will lead to successful results, or at least that the unhelpful path that you choose will take a short enough amount of time for you to change it when you discover it does not work. We want you to have productive failure that you have time to learn from.
	
	\item Write down an outline of your intended procedure. You might end up changing this as you go, but it is helpful to start with a plan and then change it, rather than having no plan at all.
	
	\item For your procedure, list the sources of uncertainty involved with each measurement. For each source, identify whether it is a random or instrumental uncertainty.
	
	\item Execute your procedure, including setup, data collection, calculation of area, uncertainty estimation and propagation.
	
	\begin{framed}
	At the end of this step, you should have a table of coffee filter cross-sectional areas, with uncertainties.
	\end{framed}
	
	\item Once you are done collecting this data, review your written procedure and correct it to match what you actually did, and ensure you have sketched any measurement setups, so you can include it in the lab report. In particular, ensure that you have enough written so you can demonstrate Rubric Rows F1, G1 and G2 in your report (see Tables~\ref{rubric:f} and \ref{rubric:g}).
	
\end{enumerate}

\subsection{How fast do the filters fall?}

\textbf{Goal:} Determine how long it takes each coffee filter to fall.

\begin{enumerate}
	\item Review Rubric B (Table~\ref{rubric:b}) and discuss any unclear expectations with your group and the instructor.
	
	\item Identify any variables (things that could change between measurements --- either between measurements of the same filter, or among different filters) that could affect the fall time other than the coffee filter's cross-sectional area. If there is controversy in the group, feel free to test what variables might affect that fall time.
	
	\item Since you are testing how the fall time is related to the filter's area, you should hold the other variables constant, so that they affect all the filters in the same way. For each variable identified in the previous step, decide how to keep that constant.
	
	\item Brainstorm different methods you could use to determine the time it takes for the filter to fall. Feel free to play with the equipment as desired. Here are some things to consider:
	\begin{itemize}
		
		\item Will you measure the fall time directly, or will you measure something else and use that to calculate the time?
		
		\item For each method you consider, there may be different sources of uncertainty --- the resolution of the measuring devices themselves, how you use them to measure, etc. If there is a source of random uncertainty, then you will need to take several measurements and use Appendix~\ref{unc:random} to determine the uncertainty.
		
		\item If you make a measurement and use that measurement in an equation to find the time, you will need to propagate uncertainty as described in Appendix~\ref{unc:sec:prop}.
	\end{itemize}
		
	\item Decide on your method and discuss it with an instructor before you begin. They will help increase the chances that your method will lead to successful results, or at least that the unhelpful path that you choose will take a short enough amount of time for you to change it when you discover it does not work. We want you to have productive failure that you have time to learn from.
	
	\item Write down an outline of your intended procedure. You might end up changing this as you go, but it is helpful to start with a plan and then change it, rather than having no plan at all.
	
	\item For your procedure, list the sources of uncertainty involved with each measurement. For each source, identify whether it is a random or instrumental uncertainty.
	
	\item Execute your procedure, including setup, data collection, calculation of area, uncertainty estimation and propagation.
	
	\begin{framed}
	At the end of this step, you should have a table of coffee filter cross-sectional areas, with uncertainties, with another column for fall time, with uncertainty in the fall time.
	\end{framed}
	
	\item Once you are done collecting this data, review your written procedure and correct it to match what you actually did, and ensure you have sketched any measurement setups, so you can include it in the lab report. In particular, ensure that you have enough written so you can demonstrate Rubric Rows B5, F1, G1 and G2 in your report (see Tables~\ref{rubric:b}, \ref{rubric:f}, and \ref{rubric:g}).
\end{enumerate}

Now that you have these measurements, it is time to find a pattern.

\subsection{Finding a pattern}

The penultimate step in an observational experiment is to find a pattern. Note that we are not explaining why this pattern is happening yet --- we are focusing on describing it first.

\textbf{Goal:} Find a pattern in the data and describe it mathematically.
\textbf{Available equipment:} Computer with spreadsheet software

\begin{enumerate}
	\item Use a plotting program, for example LibreOffice Calc or Microsoft Excel, to plot a graph of fall time vs. filter area. The independent variable should be on the horizontal axis. The axes should each be labeled with the quantity name and the unit in parentheses. For example, if you measured fall time in seconds, then the axis label should be something like ``fall time (s)''.
	
	\item In that graph, include also the uncertainty in each value. This usually involves right-clicking on a data point and selecting ``error bars''. Then you can highlight the column of cells that include the uncertainties.
	
	\item\label{filters:step:what-shape} Visually, discuss what shape the data points make. Speculate what kind of relationship you see. Is it proportional? Linear? Parabolic? Exponential? Logarithmic?
	
	\item Create a line of best fit (or ``trend line'') in the graph using the software. Choose the equation type to match what your group guessed in the previous step. If the line obviously does not match the data, try again with a different equation type. Quantitatively, the goodness of fit of a line (how close the line is to your data points) can be represented by the correlation coefficient, given as $r^2$ in the software. If $r^2 \gtrsim 0.8$, then the equation that you found describes the data fairly well. Record that equation and the $r^2$ (or RMSE if given).
	
	\item Make a final determination for describing in words the pattern found. If it applies, you should use one of the terms given in Step~\ref{filters:step:what-shape} in order to precisely describe the pattern.
	
	\item Review Rubric Rows B7 and B8 in Table~\ref{rubric:b} and ensure that you are demonstrating them here or have enough information to do so in your lab report.
\end{enumerate}

%\subsection{Finishing up}
%
%Before you leave lab, be sure that you have reviewed the 7 rubric rows that are being used to assess your report and that you are equipped to do as well on them as you would like to. The visuals that you should definitely have in your report are sketches of your measurement setups and a graph of fall time vs. filter area. You may want to have others as well, but those you will definitely need.

%\subsection{Optional Fun Time Extra}
%
%To extend the lab just a little bit and break out of the ``observational experiment'' frame, and to check to see if your mathematical pattern (line of best fit) is consistent with other physics principles, you can extrapolate down to zero cross-sectional area. As an optional part of the lab with no added benefit to your grade (but with perhaps glee for your inner scientist), you can do this.
%
%Using your equation for the relationship between the two variables, predict the freefall time of an object with zero cross-sectional area. You can compare this with the theoretical value for this time $t$, given for an object that is falling from height $h$, with the strength of the local gravitational field $g = 9.8\:$m/s$^2$, with the following formula:
%\begin{equation}
% t = \sqrt{\frac{2 h}{g}}
%\end{equation}
%How close did you get?

\section{Report checklist and grading}

Each item below is worth 10 points, and there is an additional 10 points for attendance and participation. See Appendix\ \ref{cha:lab-report-format} for guidance on writing the report and formatting tables and graphs.

\begin{itemize}
	\item Your detailed procedure, with labeled sketch, of finding the area of the filters. This  includes the experimental setup, how you collected data and estimated the uncertainty of your measurements.
	
	\item Same as above, but for finding the fall times. In particular, how did you hold the other variables, besides the area of the filter, constant?
	
	\item Plot of fall time vs. filter area (fall time on the vertical axis). Include a best fit line.
	
	\item Describe the pattern that you found, the mathematical relationship between fall time and filter area.
	
	\item Discuss the findings and reflect deeply on the quality and importance of the findings. This can be both in the frame of a scientist conducting the experiment (``What did the experiment tell us about the world?'') and in the frame of a student (``What skills or mindsets did I learn?'').
\end{itemize}