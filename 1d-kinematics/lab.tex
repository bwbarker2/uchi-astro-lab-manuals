\chapter{One-dimensional kinematics}

\section{Learning goals}

\begin{enumerate}
 \item Represent the motion of an object graphically, verbally, and mathematically. Change representations as required to describe your observations effectively.
 
 \item Apply kinematics ideas to solve a practical problem.
 
 \item Estimate experimental uncertainties in order to make judgments about experimental results.
\end{enumerate}

\section{Lab Team Roles}

Decide which team members will hold each role this week: facilitator, scribe, technician, skeptic.

\section{Observation Exp: Observe and represent motion}

\textbf{Goal:} Represent the motion of an object graphically, translate between graphical and verbal representations, and change position, velocity, and acceleration consistent with those representations. This focuses on the part of the observation experiment that involves recording data.

\textbf{Available equipment:} Simulation found at \url{https://physics.bu.edu/~duffy/HTML5/1Dmotion_graph_matching.html}

\begin{steps}
	\item Open the simulation at the address listed above. Notice that there is a position vs. time graph, a velocity vs. time graph, and a motion diagram.
	
	\item For the following different situations, imagine that you are wearing a motion sensor and it will make a motion diagram as you walk, marking your position every second. For each way of walking described, adjust the initial position, velocity, and acceleration sliders to create the motion diagram that it would create. Each way of recording your observations is called a "representation". Construct an organized table similar to the one in Table\ \ref{1dk:tab:reps} to record different representations.
	
	\begin{table}
		\begin{tabular}{p{.3\textwidth}|p{.3\textwidth}|p{.3\textwidth}}
			\textbf{verbal description of experiment} & \textbf{motion diagram} & \textbf{graphs} \\
			\hline
			Include where you start and end your motion, how you moved, and the time interval for which you moved & Remember to include dots and the number line. & position-vs-time and velocity-vs-time
		\end{tabular}
		
		\caption{Format for table of representations.}\label{1dk:tab:reps}
		
	\end{table}
	
	Try the following four experiments:
	
	\begin{enumerate}
		\item Stand still, then walk forward at a steady pace, then stand still again.
		
		\item Walk backward at a steady pace.
		
		\item Walk forward at a steady pace for 4 seconds, then walk steadily but at a faster pace backward for the next 2 seconds.
		
		\item Walk forward at a steady pace for 2 seconds, then continually increase your speed.
	\end{enumerate}
	
	\item A fellow student, Alex, does not understand why it is a good idea to represent the same information in these three different ways (verbal, motion diagrams, graphs). What do you say to Alex to help them understand the use in doing this?
	
	\item What was the purpose of this first experiment? Specifically what did you learn? How did you learn it?
	
\end{steps}

\section{Testing exp: Do you understand graphs?}

\textbf{Goal:} Adjust the simulation parameters in such a way that the graphs produced by your imagined motion match a few pre-determined graphs.

\textbf{Available equipment:} Simulation found at \url{https://physics.bu.edu/~duffy/HTML5/1Dmotion_graph_matching.html} 

\begin{steps}
	
	\item Open the simulation at the address listed above. Notice the buttons for Graph 1 through Graph 10.
	
	\item Pick three of the position graphs (Graphs 1--5) and three of the velocity graphs (Graphs 6--10) to match.
	
	\item Do the following for each:
	\begin{enumerate}
		\item Before you proceed to perform the experiment, you need to decide how you will move the sliders. Predict, based on your observations from the first experiment and your knowledge of kinematics, how you should move to produce graphs that match those that are on the screen. \textbf{Record your predictions.}
		
		\item After you have made your prediction, perform the experiment by adjusting the position, velocity, and acceleration sliders and using the play and pause buttons. \textbf{Record the outcome graph.}
		
		\item If the graph produced by your motion did not match the provided graphs, \textbf{discuss in your report} possible reasons and how you would move differently to make the motion match the graph more closely.
		
		\item Where on the graph is the position of the object represented?  Where is the time represented?
	\end{enumerate}
	
	\item How do motions that are represented by $x(t)$ lines with various slopes differ from each other? What information can we obtain from the slope of an $x(t)$ graph?
	
\end{steps}

\section{Application experiment: how long is the brick?}

\begin{itemize}
	\item \textbf{Goal:} Determine the length of a brick in a mechanics simulation.

	\item \textbf{Available equipment:} Stopwatch (app, website, watch, etc.), simulation at \url{https://phet.colorado.edu/sims/html/forces-and-motion-basics/latest/forces-and-motion-basics_en.html}

	\item \textbf{Rubrics to focus on:} D2, D4, G1, G2, G4
	
	\item Since this is the first application experiment we've done, read through the application experiment rubric, D. Rubric G is also important, as it is about data analysis, and this is a quantitative experiment.
	
	\item For comparing two values with uncertainties, to see if they are the ``same'' or not, see Appendix~\ref{unc:sec:comparing}.
\end{itemize}

\begin{steps}
	\item Open the simulation in the link above and go to the section marked "Motion". Notice that you can select different view options in the upper right hand corner.
	
	\item Brainstorm two \textit{independent} experiments to determine the length of the brick. One of the ways must use the formula that describes the relationship between displacement $\Delta x$, velocity $v$, and time $t$ during constant velocity,
\begin{equation}
	 \Delta x = v t \,.
\end{equation}
	
	Another way could involve the typical size of other objects in the simulation.
	
	\item When you come up with at least 2 possible experiments, \textbf{contact an instructor or TA} and discuss your experiments with them.
	
	\item For each experiment you conduct, including the following in your report:
	
	\begin{enumerate}
		\item Describe your experimental procedure. Include a sketch of your experimental design.
		
		\item List the sources of experimental uncertainty.
		
		\item Explain what steps you will take to minimize experimental uncertainty.
		
		\item Perform the experiment. Record the data using appropriate representations (motion diagrams, tables, graphs, etc.). Determine the uncertainty in the measurements you made (you will need to do multiple trials to do this).
		
		\item Use your measurements and their uncertainties to determine the length of the brick. Propagate uncertainties through your calculations as described in Appendix~\ref{unc:sec:prop}.
		
		\item Report your result as a value plus or minus the uncertainty (e.g. $5 \pm 1$ m).
	\end{enumerate}

	\item Compare the two values you obtained for the brick length using the method described in Appendix~\ref{unc:sec:comparing} and obtain a $t'$ value.
	
	\item Taking into account experimental uncertainties and the assumptions you made, decide if these two values are consistent or not. If they are not consistent, explain possible reasons for how this could have happened (for example, assumptions made, underestimating uncertainty).
	
	\item Decide on a final value (including uncertainty of that value) for the brick length based on the results of your experiments.

	\item Describe the shortcomings you noticed in the experiments. Suggest specific improvements.
	
\end{steps}

\section{Why did we do this lab?}

\begin{steps}
	\item Explain how each of the group member's understanding of representations of motion is different now compared to before the lab.
	
	\item In this lab you conducted an observation experiment and an application experiment. How are the purposes of an observation experiment and an application experiment different?
	
	\item In your own words, what is it meant by "experimental uncertainty"?  Why should we care about it?
	
	\item Give real-life examples of instances when people need to collect and analyze quantitative data (numbers with units) to describe and understand what is happening.
\end{steps}

\section{Group dynamics}

\begin{steps}
	\item Write a 100--200 word reflection on group dynamics and feedback on the lab manual. Address the following topics: who did what in the lab, how did you work together, what successes and challenges in group functioning did you have, and what would you keep and change about the lab write-up?
	
	\item Write a paragraph reporting back from each of the four roles: facilitator, scribe, technician, skeptic. Where did you see each function happening during this lab, and where did you see gaps?
\end{steps}

\section{Report checklist and grading}

The lab grade consists of 3 points for each of seven scientific ability rubric rows (the 5 listed above, which apply just to that section, as well as F1 and F2, applied to the entire report), and 9 points for providing evidence in the lab report of completing all steps of the lab, including answering every question, for a total of 30 points.