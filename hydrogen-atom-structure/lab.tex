\chapter{Discovering the Hydrogen Atom}

\section{Learning Goals}

\begin{itemize}
	\item Learn to critically evaluate model of physical phenomena and understand how these models are formulated.
	
	\item Develop methods for testing and working with objects which cannot be directly observed. 
	
	\item Gain an understanding of the atomic model and why it differs from what classical physics predicts.
\end{itemize}

\section{The Building Blocks of Matter}

The word atom finds its roots in the Greek word \textit{atomos} which means indivisible and is believed to have first been used by Democritus to refer to the indivisible spheres which he believed to be the building blocks of our world. Today, we have a far better understanding of the atom and its structure, however, this is information we take for granted. For millennia, there was no clear answer as to what the smallest unit of everything was, and only relatively recently did we develop tools which allowed us to study them more closely. The first comprehensive atomic model was developed in 1803 by John Dalton, who thought of them as solid spheres which changed depending on the element they made up. Then in 1904 JJ Thompson developed the ``Plum Pudding'' model, which depicted the atom as made up of electrons floating within a cloud of positive charge. Then in 1911 Ernst Rutherford proposed the nuclear model, where the positive charge was concentrated in the center of the atom. Two years later, Niels Bohr proposed that the orbits were fixed. Finally Erwin Schrodinger proposed in 1926 the quantum model, where electrons exist around the nucleus in a ``cloud of probability''. It might seem that as these models progress, the less and less intuitive they become. This is largely because over time the normal physics we observed became less and less useful at explaining the effects we observed. However, by the end of this lab, hopefully you will have a better understanding of the intuition behind our model of the atom.  

\section{Lab Team Roles}

Decide which team members will hold each role this week: facilitator, scribe, technician, skeptic. If there are three members, consider having the skeptic double with another role. Consider taking on a role you are less comfortable with, to gain experience and more comfort in that role.

Additionally, if you are finding the lab roles more restrictive than helpful, you can decide to co-hold some or all roles, or thinking of them more like functions that every team needs to carry out, and then reflecting on how the team executed each function.

\section{Model Evaluation and Construction} 

\subsection{Goal} 
When thinking of an atom, we usually tend to imagine a series of electrons orbiting around a nucleus in nice circular orbits. While at first this model might seem unremarkable, the reality is that the physics that describes it is a lot weirder than you think! In order to see why this is, lets see if we can reconstruct the model of the atom qualitatively first.

\subsubsection{Rubric Rows to Be Assessed}

B9, C4, C5, C6 D1


\begin{steps} 
	\item First, lets take the classical planetary model of the atom at face value. Using only classical physics concepts, try and find whats wrong with this model. 
	\begin{itemize} 
		\item Think first about what happens to satellites orbiting Earth as they interact with the atmosphere. What forces are acting on it? What happens to it over time?
		
		\item Consider that changing the orbital radius of a body requires a corresponding change in energy. Think about the changes in kinetic and potential energy as the satellite moves closer and farther from the Earth. 
	\end{itemize}
	\item Once your group has come up with an answer, describe what happens to the orbiting object in terms of its energy and its position over time. What implications does this have when you apply it to the atom?
	
	\item After identifying the problem with the model, develop several ideas for a new model which solves the issues you found. \textbf{Forget any previous knowledge of the atom. Think outside the box. Write down your ideas and describe how they solve the problem.} 
	\begin{itemize}
		\item These models don't have to be too complicated, just try thinking of different way in which the electron might move or behave in relation to the nucleus.
	\end{itemize}
\end{steps}

\section{Observation Experiment: How does an atom absorb and release energy?}

\subsubsection{Goal:}
Now that you have a rough model of the atom, lets supplement that with some observations to see if you can come create a more robust mode. The difficulty with trying to learn about the structure of the atom is that you have no way of directly observing it. However, you are able to interact with it and observe the results. In this experiment, you will be firing photons at an atom and you will attempt to modify your model of the atom based on your observations.
\subsubsection{Available Equipment}

\begin{itemize}
	\item The PHET Lasers simulation: \url{https://phet.colorado.edu/en/simulation/legacy/lasers}
\end{itemize} 

\subsubsection{Rubrics to be assessed}

B1, B4, B7, G5, B9

\subsubsection{Steps}

\begin{steps} %clarify that model and hypothesis and model are pretty much the same. Work on the wording. 
	\item When you enter the simulation, on the right-hand panel select the option for "three" under the "Energy Levels" heading.
	
	\item Take a moment to familiarize yourself with the different parameters you can control. In particular focus on the lamp controls and the energy level controls. 
	
	\item In your group, describe any patterns you observe in the absorption and emission of photons. Be precise and include a visualization of the emission spectra. Try and estimate the wavelengths of the emitted photons. 
	
	\item Once you believe you have found a pattern, incorporate it into your atomic model and describe how it responds to this increase in energy
	\begin{itemize}
		\item Think about where the energy goes in the atom once its absorbed
		
		\item Which particle in the atom do you think would have the most freedom?
		
		\item Does your atom change in configuration following an increase in energy? 
	\end{itemize}

	\item Describe what, if any, modifications you made to your model to reflect this observed phenomenon. What was your reasoning behind changing the model or leaving it the same? Write down your explanation.  
\end{steps}

\section{Testing Experiment: Black-boxing the Hydrogen Atom}


\subsection{Available Equipment}

\begin{itemize}
	\item PHET Models of the Hydrogen Atom Lab: \url{https://phet.colorado.edu/en/simulation/legacy/hydrogen-atom}
\end{itemize}

\subsection{Rubrics to be assessed}

C1, C4, C5, C7

\subsection{Goal}
Now that you have worked to develop your own theoretical model of the atom, you will see how they compare to real life theoretical models proposed for the atom. You will be given a black-box, behind which is a hydrogen atom. 
\subsection{Steps}

\begin{steps}
	\item At first, leave the simulation in "Experiment Mode" and familiarize yourself with the available controls. 
	
	\item Notice the spectrometer which allows you to see the wavelengths of light emitted by the atom. Given this set up, what does your model predict would happen? Does it predict an order of emission? Write down a qualitative description what would happen according to your model. 
	
	\item Using the tools at your disposal, determine your model's predictive power. Describe in what ways your model failed or succeeded at describing the pattern you observe.
	
	\item In the previous experiment, you saw how atoms can only absorb and emit a very specific amount of energy. Provide a rough estimate of the different amounts of energy which this atom can absorb
	\begin{itemize}
		\item Remember that the color of light is determined by its frequency, and the higher the frequency of an object, the more energy it has. 
		
		\item Use the equation $\mathit{E} = \mathit{h}\nu$ where $\mathit{E}$ is energy joules, $\mathit{h}$ is planck's constant equal to \newline $6.62 \times 10^{-34}$Js, and $\nu$ is the frequency of the emitted photon in hertz. 
	\end{itemize}
\end{steps}

%Seperate into different experiment. describe what each model predicts in words, say qualitatively how close it is to experiment. For each of the models why would somebody find these to be reasonable predictions

\section{Model Analysis}

\subsection{Available Equipment}

\begin{itemize}
	\item PHET Models of the Hydrogen Atom Lab: \url{https://phet.colorado.edu/en/simulation/legacy/hydrogen-atom}
\end{itemize}

\subsection{Goal}
So far, you have been developing your own model of the atom and determining its predictive power. Now you will analyze different atomic models proposed and you will attempt to intuit the reasoning behind them. 

\subsubsection{Rubric items to be assessed during the experiment}

D3, G4, C6, C5

\begin{steps}
	\item Using the toggle in the top-left corner of the simulation, switch to "Prediction" mode. You will find a list of atomic models organized from "classical" to "quantum". 
	
	\item With your group, go through each model and write down a qualitative description of the model. What does each model predict? How does it solve the problems you found in the earlier sections? Why is each model a reasonable prediction of the atom? Make sure to write down your descriptions
	
	\item Now that you have a description of each model, try and determine which model predicts an emission pattern closest to the black-boxed atom in "experiment mode". Include screenshots of the emission spectra.
	
\end{steps} 

\section{Report checklist and grading}
