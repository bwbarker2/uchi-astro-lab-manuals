\chapter{Measuring the distance to a galaxy using globular clusters}

%todo add lab activity to measure differential light intensity at different distances from light bulbs
%todo clarify that M5 (rather than Sun) should be used to compare to M87 cluster

%In this lab, you will measure the distance to the Virgo galaxy, which we will use for next week’s laboratory as the first rung in our distance ladder measuring the distance to other galaxies.

Measuring the distance to another galaxy is not straightforward, since it would take too long to fly there, and it would take too long to bounce light off of it (among other difficulties). In this lab, you will use the apparent brightness of star clusters orbiting another galaxy to estimate the distance. After all, things that are further away from us appear dimmer.

In this lab, we will use a globular cluster in the Milky Way called M5 (see Figure~\ref{gc:fig:m5}) as the first step of our distance
ladder to other galaxies. First, we will compare the brightness of the Sun to the brightness of a sun-like star in the globular cluster M5 (which is in the Milky Way and close enough that we can resolve individual stars) to find the distance to M5. Then we will compare the brightness of the entire M5 cluster to a cluster orbiting the galaxy M87 (that cluster is so far away that we see it as just one bright mass). This will allow us to find the distance to that galaxy! Ideally, we should compare many globular clusters in the Milky Way to
M87, here we will use just one cluster, which is enough to demonstrate the principle.

\begin{framed}
	\textbf{Confusing nomenclature alert!}
	
	M5 is a globular cluster in our galaxy, while M87 is an entire galaxy. They share the prefix `M', even though they are very different objects, because these numbers are part of the Messier catalogue, a numbered list of astronomical objects seen in the sky by Charles Messier, who was searching for comets and wanted to record these objects so that when he saw them during his comet search, he could ignore them. So don't forget to consider the negative space, and the work you do along the way!
\end{framed}

\begin{figure}
	\centering
	\includegraphics[width=0.7\textwidth]{globular-cluster/potw1118a}
	\caption{Messier 5 (M5) is a globular cluster (a gravitationally bound collection
		of stars) of more than 100,000 stars in the Milky Way Galaxy. Located at Right
		Ascension (RA) = 229.640\textdegree, and Declination (Dec) = 2.075\textdegree.
		The above
		image is 2.85 arcmin on a side, or about 1/20th of a degree.
		Image source: ESA/Hubble \& NASA, \url{http://www.spacetelescope.org/images/potw1118a/}}\label{gc:fig:m5}
\end{figure}

\section{Team roles}

\textbf{Decide on roles} for each group member. The available roles are:

\begin{itemize}
	\item Facilitator: ensures time and group focus are efficiently used
	\item Scribe: ensures work is recorded
	\item Technician: oversees apparatus assembly, usage
	\item Skeptic: ensures group is questioning itself
\end{itemize}

These roles can rotate each lab, and you will report at the end of the lab report on how it went for each role. If you have fewer than 4 people in your group, then some members will be holding more than one role. For example, you could have the skeptic double with another role. Consider taking on a role you are less comfortable with, to gain experience and more comfort in that role.

Additionally, if you are finding the lab roles more restrictive than helpful, you can decide to co-hold some or all roles, or think of them more like functions that every team needs to carry out, and then reflecting on how the team executed each function.

\subsection{Add members to Canvas lab report assignment group}

\begin{steps}
	\item On Canvas, navigate to the People section, then to the ``Groups'' tab. Scroll to a group called ``L2 Cluster [number]'' that isn't used and have each person in your group add themselves to that same lab group.
\end{steps}

This enables group grading of your lab report. Only one person will submit the group report, and all members of the group will receive the grade and have access to view the graded assignment.

\section{Brightness and distance}

If two identical lightbulbs are placed with one close to you and another farther away, the more distant one will appear dimmer. This is because the light from a spherical emitting source spreads out over a spherical shell that gets larger as the light gets more distant from the source. So if sources 1 and 2 have the same luminosity, then their distances $d$ and apparent brightness
$b$ are related by
\begin{equation}
\frac{b_1}{b_2} = \left( \frac{d_2}{d_1} \right)^2 \,.
\end{equation}
Since the numbers we will extract from the images are either in brightness or magnitudes,
it is convenient to re-cast this relation in terms of magnitudes. Magnitudes $m_1$ and $m_2$ are
related to brightness $b_1$ and $b_2$ by
\begin{equation}
m_2 - m_1 = 2.5 \log \left[ \left( \frac{b_1}{b_2} \right)^2 \right] \,.
\end{equation}
Combining the two equations, we get
\begin{equation}\label{gc:eq:m-d}
 \log(d_1/d_2) = 0.2 (m_1-m_2) \,.
\end{equation}
This says that once we have measured magnitudes $m_1$ and $m_2$ for two sources, then we can
derive the ratio of their distances from us, \textit{as long as they have the same luminosity}.

\section{Road Map}

To keep track of the steps in this lab, we will fill in Table~\ref{gc:tab:mag}. In this table, the entry for the magnitude of M5 refers to the sum of all of its stars. In
principle, we could measure this ourselves with the roof-top telescope, but for this lab we take
a value from a catalog of such data. The SDSS data cannot be used because the stars
are too crowded together for an accurate measurement.

\begin{table}
	\centering
	\begin{tabular}{l|c|c}
		\toprule
		\textbf{Object} & \textbf{Magnitude} & \textbf{Distance (AU)} \\ \midrule
		Sun & -26.89 & 1 \\\midrule
		average Sun-like star in M5 &  & \\\midrule
		M5 itself & 5.65 & \\\midrule
		average M87 globular cluster & &
		\\ \bottomrule
	\end{tabular}
	\caption{Table of magnitude and distance.}\label{gc:tab:mag}
\end{table}

The first step is to make a \textit{color-magnitude diagram} for the stars in M5 to find a star that
has similar to the Sun; we assume that such a star has the same luminosity as the Sun. The
magnitude of the star (specifically its $r$-band magnitude) gets entered into the above table,
and you derive the distance to M5.

The second step is to identify faint things surrounding the galaxy M87 that are likely to be
globular clusters associated with it, and get their magnitudes (again the $r$-band magnitude)
from the database. Some value that properly represents the ensemble gets entered into the
above table and you derive the distance to M87 by comparing the magnitudes of M5 and M87.

\textit{To summarize:} the distance to the M87 galaxy depends on two assumptions: 1) stars
with Sun-like colors in the globular cluster M5 have the same luminosity of the Sun. 2)
Globular clusters like M5 in the Milky Way have luminosities that are comparable to
the globular clusters in M87. Neither of these assumptions is necessarily well justified based on information available to you, but there are checks that reassure us that the assumptions
are good enough for at least a first estimate of distance.

\section{Analyzing the M5 globular cluster}

First you'll retrieve from an online database the magnitude of stars in the region of sky where M5 is. In the window at \url{http://skyserver.sdss.org/dr13/en/tools/search/sql.aspx}, enter the following query:

\begin{verbatim}
SELECT TOP 200
   objid,ra,dec,u,g,r,i,z
FROM Star
WHERE
   r BETWEEN 10 AND 23
   AND ra between 229.50 and 229.78
   AND dec between 2.2 and 2.3
\end{verbatim}

\subsection{Questions and results for your report}

\begin{steps}
	\item From the data above, create a .csv file, rename it M5.csv. Read it into
	a spreadsheet, and make columns for the colors $g - r$, $r - i$, and $g - i$. The Sun has
	colors $g - r = 0.44$, $r - i = 0.11$, and $g - i = 0.55$. Plot the $r$ magnitude vs. one of the colors (e.g.
	$g - r$), and reverse the $r$ magnitude axis, since lower magnitudes represent brighter objects. On this color-magnitude diagram, identify
	the \textit{main sequence} of stars. This plot is called a color-magnitude diagram, which is similar to an H-R diagram as seen in Figure~\ref{gc:fig:hr}.
\end{steps}

\begin{figure}
	\includegraphics[width=\textwidth]{globular-cluster/eso0728c}
	\caption{Herzsprung-Russell (H-R) diagram, plotting stars according to their luminosity and surface temperature. Luminosity is related to magnitude, and surface temperature is related to color. Image Source: ESO (\url{https://www.eso.org/public/images/eso0728c/}}\label{gc:fig:hr}
\end{figure}

\begin{steps}
	\item Begin to fill out Table~\ref{gc:tab:mag} with the magnitude and distance to a Sun-like star
	in M5. When you finish this lab and turn in this lab report, this table will
	be completely filled out.
\end{steps}

\section{Analyzing the globular clusters near the M87 galaxy}

Figure~\ref{gc:fig:m87} shows the field surrounding the giant Virgo galaxy M87, also known as NGC 4486.

\begin{figure}
	\centering
	\includegraphics[width=0.7\textwidth]{globular-cluster/eso1525a}
	\caption{Messier 87 (M87) is a nearby elliptical galaxy in the constellation Virgo. It is
		known for having a large population ($\sim 10,000$) of globular clusters, about 100 times more
		than the Milky Way Galaxy. Centered at RA=187.706\textdegree and Dec=12.391\textdegree, the above image
		is 97 arcminutes across. Image source: Chris Mihos (Case Western Reserve University)/ESO, \url{http://www.eso.org/public/images/eso1525a/}.}\label{gc:fig:m87}
\end{figure}

The task is to find the magnitudes for the faint speckles surrounding M87 that are barely visible in
Figure~\ref{gc:fig:m87}, namely its globular clusters. We set up a similar query to that used for M5, except
of course the coordinates (RA, Dec) are different.

\begin{steps}
	\item Enter the following query in SkyServer:

\end{steps}

\begin{verbatim}
SELECT TOP 200
   objid,ra,dec,u,g,r, i,z
FROM Star
WHERE
   r BETWEEN 10 AND 23
   AND ra between 187.591 and 187.821
   AND dec between 12.278 and 12.504
\end{verbatim}


The globular clusters are so far away that each cluster of stars appears as, and is categorized as, stars in
the SDSS database.

\begin{steps}
	\item As a cross-check, also run the above query in a random piece of sky at least
5\textdegree{} away.
\end{steps}

\subsection{Questions and results for your report}

\begin{steps}
	\item Make color-magnitude diagrams for both samples (M87 and random-sky)
	and compare them. Does either color-magnitude diagram show any
	evidence for a correlation between brightness (magnitude) and color for the
	plotted points?
	
	\item Once you have identified which of the sources on your M87 color-magnitude
	diagram can be identified with a population of globular clusters surrounding
	M87, argue which apparent magnitude should be selected to enter into the table, and do so. Why is there a range of magnitudes? How would you make this
	process more precise? What other sources of uncertainty do you think
	there are?
	
	\item Calculate the distance to M87 using Equation~\ref{gc:eq:m-d}, comparing its magnitude to that of M5 (the entire cluster M5, not the individual star you used earlier). If you have not already converted AU to
	parsecs, do so now to get the distance in mega-parsecs ($1\:\mathrm{Mpc} = 2.06 \times
	10^{11}\:\mathrm{AU}$). The accepted value for the distance to the Virgo cluster (the galaxy cluster that M87 is in) is 16.4
	Mpc. From your uncertainties above, how well do these two values agree within your expected level
	of uncertainty? See Appendix~\ref{unc:sec:comparing} for details of how to determine this.
	
	\item Based on the distance you found, for the light that arrived at the sky survey's telescope from M87, when was it emitted? What was happening on Earth at that time?
\end{steps}

\section{Report checklist and grading}

Include the following in your lab report. See Appendix~\ref{cha:lab-report-format} for formatting details. Each item below is worth 10 points.

\begin{enumerate}
	\item Completed table of magnitudes and distances
	
	\item Color magnitude diagram for M5
	
	\item Calculation and determination of magnitude and distance of a Sun-like star in M5 (Step 2)

	\item Color-magnitude diagrams for M87 and random-sky
	
	\item Questions in Steps 5--6
	
	\item Calculation and determination of magnitude and distance of M87, with analysis of uncertainty and comparison (Step 7)
	
	\item Age of light and what was happening on Earth (Step 8)
	
	\item Discuss the findings and reflect deeply on the quality and importance of the findings. This can
	be both in the frame of a scientist conducting the experiment (“What did the experiment tell us
	about the world?”) and in the frame of a student (“What skills or mindsets did I learn?”).
	
	\item Write a 100--200 word paragraph reporting back from each of the four roles: facilitator, scribe, technician, skeptic. Where did you see each function happening during this lab, and where did you see gaps? What successes and challenges in group functioning did you have? What do you want to do differently next time?
		
\end{enumerate}