\chapter{Analysis of Uncertainty}

A physical quantity consists of a value, unit, and uncertainty. For example, ``$5 \pm 1\,$m'' means that the writer believes the true value of the quantity to most likely lie within 4 and 6 meters\footnote{The phrase ``most likely'' can mean different things depending on who is writing. This is discussed in more detail in Section\ ???.}. Without knowing the uncertainty of a value, the quantity is next to useless. For example, in our daily lives, we use an implied uncertainty. If I say that we should meet at around 5:00 pm, and I arrive at 5:05 pm, you will probably consider that within the range that you would expect. Perhaps your implied uncertainty is plus or minus 15 minutes. On the other hand, if I said that we would meet at 5:07 pm, then if I arrive at 5:10 pm, you might be confused, since the implied uncertainty of that time value is more like 1 minute.

Scientists use the mathematics of probability and statistics, along with some intuition, to be precise and clear when talking about uncertainty, and it is vital to understand and report the uncertainty of quantitative results that we present.

\section{Significant digits}

In science classrooms, the notion of \textbf{significant digits} (also called significant figures) is a formalized way of notating implied uncertainty, as well as a crude way of combining uncertainties (or ``propagating'' them) when using uncertain quantities in calculations.

In general, 

\section{Propagation of Uncertainty}

When we use an uncertain quantity in a calculation, the result is also uncertain. To determine by how much, we give some simple rules for basic calculations, and then a more general rule for use with any calculation which requires knowledge of calculus.

If the measurements are completely independent of each other, then for quantities $a \pm \delta a$ and $b \pm \delta b$, we can use the following formulas:
\begin{equation}\label{unc:add}
\textrm{For } c = a + b \textrm{ (or for subtraction), } \delta c = \sqrt{(\delta a)^2 + (\delta b)^2}
\end{equation}

\begin{equation}\label{unc:mult}
\textrm{For } c = ab \textrm{ (or for division), } \frac{\delta c}{c} = \sqrt{\left(\frac{\delta a}{a}\right)^2 + \left(\frac{\delta b}{b}\right)^2}
\end{equation}

\begin{equation}\label{unc:exp}
\textrm{For } c = a^n,\, \frac{\delta c}{c} = n \frac{\delta a}{a}
\end{equation}

If you are familiar with calculus, you may want to use this general formula for the uncertainty $\delta f$ of a function $f$ of $N$ independent values $x_i$, each with uncertainty $\delta x_i$:
\begin{equation}\label{unc:general}
\delta f = \sqrt{ \sum_{i=1}^{N} \left(\frac{\partial f}{\partial x_i} \delta x_i\right)^2 } \, .
\end{equation}
Notice that Eqs.\ \ref{unc:add} through \ref{unc:exp} can be derived from Eq.\ \ref{unc:general} for those specific cases.

\subsubsection{How many digits to report?}

After even a single calculation, a calculator will often give ten or more digits in an answer.
For example, if I travel $11.3 \pm 0.1\:$km in $350 \pm 10\:$s, then my average speed will be the distance divided by the duration. Entering this into my calculator, I get the resulting value ``\texttt{0.0322857142857143}''.
Perhaps it is obvious that my distance and duration measurements were not precise enough for all of those digits to be useful information.
We can use the propagated uncertainty to decide how many decimals to include.
Using the formulas above, I find that the uncertainty in the speed is given by my calculator as ``\texttt{9.65683578099600e-04}'', where the `\texttt{e}' stands for ``times ten to the''.
I definitely do not know my uncertainty to 14 decimal places.
For reporting uncertainties, it general suffices to use just the 1 or 2 most significant digits (left-most digits that are not merely used as placeholders), unless you have a more sophisticated method of quantifying your uncertainties.
So here, I would round this to 1 significant digit, resulting in an uncertainty of $0.001\:$km/s.
Now I have a guide for how many digits to report in my value.
Any decimal places to the right of the one given in the uncertainty are distinctly unhelpful, so I report my average speed as ``$0.032 \pm 0.001\:$km/s''.
You may also see the equivalent, more succinct notation ``$0.032(1)\:$km/s''.

\subsubsection{What if there is no reported uncertainty?}

Sometimes you'll be calculating with numbers that have no uncertainty given.
In some cases, the number is exact.
For example, the circumference $C$ of a circle is given by $C = 2 \pi r$. Here, the coefficient, $2\pi$, is an exact quantity and you can treat its uncertainty as zero.
If you find a value that you think is uncertainty, but the uncertainty is not given, a good rule of thumb is to assume that the uncertainty is half the least significant digit (the right-most digit that is not merely used as a placeholder to identify the decimal places).
So if you are given a measured length of $1400\:$m, then you might assume that the uncertainty is $50\:$m.
This is an assumption, however, and should be described as such in your lab report.