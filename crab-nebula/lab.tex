\chapter{Origin of the Crab Nebula}

The Crab Nebula was discovered in 1731 by John Bevis, an English doctor, electrical researcher, and astronomer. The French astronomer Charles Messier found the object while searching for comets. He decided to start cataloging these fuzzy objects, thus the Crab Nebula being marked as ``M1''. In 1921, Carl Otto Lampland, a first-generation Norwegian American immigrant astronomer, discovered that the nebula was changing its structure over time.

In this lab, you will investigate how the Crab Nebula is changing and what it means for its history.

\section{Learning goals}

\begin{itemize}
	\item Identify and use analysis skills and tools to answer an astronomical question.
	
	\item Become familiar with the Crab Nebula.
\end{itemize}

\section{Lab Team Roles}

Decide which team members will hold each role this week: facilitator, scribe, technician, skeptic. If there are three members, consider having the skeptic double with another role. Consider taking on a role you are less comfortable with, to gain experience and more comfort in that role.

Additionally, if you are finding the lab roles more restrictive than helpful, you can decide to co-hold some or all roles, or thinking of them more like functions that every team needs to carry out, and then reflecting on how the team executed each function.

\section{How is it changing?}

\subsection{Goal}

Describe qualitatively and quantitatively how the Crab Nebula is changing over time.

\subsection{Available equipment}

\begin{itemize}
% not sharp enough for me
%	\item Recent images of the Crab Nebula (M1) taken by our robotic telescope at Stone Edge Observatory. These are available on Canvas in the Lab module, identified by the FITS files starting with ``M1''. Specifically use the broadband filters (g, i, r), not the narrowband filters.
	
	\item Historical images of the Crab Nebula from the STScI Sky Survey, found at \url{https://archive.stsci.edu/cgi-bin/dss_form}.
	
	\item Any software or other tools used in previous labs.
\end{itemize}

\subsection{Steps}

\begin{steps}
	\item Navigate to the URL above.
	
	\item To get the object's coordinates, type in ``M1'' in the Object Name field and select ``GET COORDINATES''. The Target name ``MESSIER 001'' should appear under the ``Retrieve an Image'' section.
	
	\item Each of the selections in the ``Retrieve from'' menu is from a different sky survey and filter. Different surveys were done at different times.
	
	\item Download a FITS file from two different surveys for the same filter.
	
	\item Open these files in DS9 and see if you notice any change in shape or structure between these (you might not see any difference). \textbf{Record your observations.}
\end{steps}

In fact, over time, it was found that the nebula was expanding, increasing in size over time. If this is true, then at some time in the past, the nebula would have been its smallest size. In fact, we know that this nebula is actually the remnant of a supernova --- an exploding star!

\begin{steps}
	\item Design an experiment to determine how fast the nebula is expanding (in arcsec/year), and how long ago the star went supernova. Discuss your experiment with a TA. Once you have decided on your experiment, \textbf{record your experimental procedure and analysis plan}.
	\begin{itemize}
		\item Detailed information about the image, including the plate scale, time of observation, and pixel size, can be found in the FITS header, viewable in DS9 with File $>$ Display Header...
		
		\item When comparing images, it can be useful to identify structures that have the same shape in each image.
	\end{itemize}
	
	\item Conduct your experiment, collecting the data, taking relevant screenshots to show what you are doing.
	
	\item Analyze your data to determine the speed of expansion and year in which the star exploded. \textbf{Record this analysis and your results with uncertainties}.
\end{steps}

Once astronomers calculated when the star had exploded, they looked back into historical records and found that indeed, astronomers from around the world (China, Japan, Iraq, North America) had observed and recorded this new or ``guest'' star. You can read about even more historical observations at the Wikipedia page for SN 1054.

\section{Report checklist and grading}

Each item below is worth 10 points. See Appendix\ \ref{cha:lab-report-format} for guidance on writing the report and formatting tables and graphs.

\begin{enumerate}
	
	\item Initial observation notes (Step 5)
	
	\item Detailed description of experimental design, including data collection and analysis (Step 6)
	
	\item Data collection results (Step 7)
	
	\item Analysis of data with final speed of expansion (in arcsec/year) and year of supernova, both with uncertainties (Step 8)
	
	\item Discuss the findings and reflect deeply on the quality and importance of the findings. This can
	be both in the frame of a scientist conducting the experiment (“What did the experiment tell us
	about the world?”) and in the frame of a student (“What skills or mindsets did I learn?”).
	
	\item A 100–200 word reflection on group dynamics and feedback on the lab manual. Address the
	following topics: who did what in the lab, how did you work together, what successes and
	challenges in group functioning did you have, and what would you keep and change about the
	lab write-up?
	
	\item Write a paragraph reporting back from each of the four roles: facilitator, scribe, technician,
	skeptic. Where did you see each function happening during this lab, and where did you see
	gaps?
\end{enumerate}